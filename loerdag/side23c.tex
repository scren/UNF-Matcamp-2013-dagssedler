\begin{minipage}[t]{100mm}
\section*{Ordforklaringer}
I ugens løb har i mødt mange nye ord, og for en sikkerheds skyld vil vi i denne artikel minde jer om hvad de betyder:
\begin{itemize}
\item \emph{Sætning} Et matematisk udsagn der er vigtig nok, til at det forventes at i kan skrive det korrkt ned på en tavle klokken fire om natten på en julefrokost.
\item \emph{Lemma} En matematisk sætning der er så let at vise, at endda mindreårige tyske får kan finde ud af at gøre det.
\item \emph{Proposition} En sætning der er er så smuk, at man frier til sin læsemakker og bliver gift af den alvidende bibliotekar på Matematisk bibliotek.
\item \emph{Korollar} En sætning der følger direkte fra en anden sætning, uden at man behøver nedskrive et bevis, hvis bare man får femten rigtige smarte idder i træk for at se at den faktisk følger af sætningen.
\item \emph{Simpel observation} Et matematisk udsagn som sandsynligvis er rigtig, men så besværligt at tjekke i detaljer at underviseren ikke gider gennemgå beregningerne.
\item \emph{Fakt} Et ikke-åbenlys, men meget vigtig matematisk udtryk, som ingen gider huske udenad.
\item \emph{Spørgsmål} Udsagn ingen endnu har (mod-)bevist, som en matematiker skrev på et serviet under en aftensmad på instituttet.
\item \emph{Standardbesvarelse} Bevis med mange fejl der udleveres uden at blive gennemgået, så fejlen først opdages når der læses til eksamen.
\item \emph{Formodning} Udsagn ingen endnu har (mod-)bevist, som en berømt matematiker skrev på et serviet under en aftensmad på instituttet.
\item \emph{Opgave} Formodning som lyder let nok, til at man kan stille dem til øvelsestimer, fordi det giver underviseren en god undskyldning for ikke at have forberedt en standardbesvarelse.
\item \emph{Skriftlig eksamen} Opgaver, hvor det ikke er åbentlyst at ingen på instituttet kan løse dem.
\item \emph{Mundtlig eksamen}  Opgaver, hvor det er åbentlyst at ingen på instituttet kan løse dem.
\item \emph{Læseferie} Det modsatte af ferie
\item \emph{Ferie} $\emptyset$
\item \emph{Forelæsning} Lokale i hvilket én person snakker og mange personer sover.
\item \emph[Øvelser} Lokale i hvilket alle tilstedeværende ville foretrække at sove-
\item \emph{Møde} Lokale i hvilket alle tilstedeværende sover
\end{itemize}

\end{minipage}

