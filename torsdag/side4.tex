\begin{minipage}[t]{180mm}
\fcolorbox{black}{white}{
\begin{minipage}[b]{30mm}
\includegraphics[width=0.5\linewidth]{unflogo.pdf}
\end{minipage}
\begin{minipage}[b]{100mm}
\Huge \textbf{UNF NEWZ} \\
\Large -- jyder er også hvirveldyr!
\end{minipage}
\begin{minipage}[b]{50mm}
\Large Søndag 15.07.2013 \\
\normalsize Redigeret i \LaTeX\ af \\ SOM, MGS, KUM, TAL, JAM
\end{minipage}
}
\end{minipage}



\begin{minipage}[b]{0.95\linewidth}
\begin{minipage}[t]{0.47\textwidth}
\vspace{1mm}
\section*{Ikke-kommutativ studenterrådgiv.}
\emph{Kære brevkasse}

\emph{Det afhænger af situationen}

\emph{Hilsen, en trold fra fysik}

\subsection*{Spørgsmål}

Kære trold

Det er et meget passende svar på mit spørgsmål, som du ved et basisskift i Minkowskirummet selvfølgelig forudså.

Som du ved har jeg i flere døgn overvejet, hvor mange kager der skal bages for at opveje den skade en én kubikmeter stor rubik's terning der falder ned fra fysiktårnet og rammer en navneskiltsansvarlig i håret. Normalt vil jeg selvfølgelig gå ud fra at skaden er med i den ansvarliges daglige kvote af ulykker, og derfor ikke udløser bonuskage i henhold til arrangøøroverenskomsten. Dog går de almindelige campmiljøregler kun ud fra, at personer rammes af højder op til 4. sal i E-bygningen på HCØ, og denne er tydelig lavere end fysiktårnet, så her ligger der et problem i forhold til den sagkyndige skøn. Til gengæld er rubik's terninger eksplicit nævnt i reglerne, og da vi ved at det kun drejer sig om en $3 \times 3 \times 3$-terning, ville man forvente at den kun gav merit for denm mindste skadekategori, som modsvares af et halvt chokoladekiks. For at opretholde proportionalitetsprincippet, skal vi dog tage højde for at denne terning er lidt større end hvad vi normalt forventer. En lineær skalering af kagen, vil for den uforberedte betragter måske virke rimeligt, men dette er ikke den passende strategi både af praktiske og teoretiske grunde. På den ene side vil det, på grund af pågående byggeri på den gamle Lillebæltsbro være problematisk at rekvirere en kvart kubikmeter kage fra Fyn, som traditionelt er det eneste sted hvor gale formænd laver gale kager til UNFs interne forbrug. På den anden side følger vi en politik om at vores beslutninger gerne skal virke fremadrettet og da vi frygter at imperiet bygger en rubik's terning i dødsstjernestørrelse og vi ikke vi ikke har budgetteret med en ostekage i månens størrelse, må vi være tilbageholdende med at skalere lineært.

Så, kære trold, hvor meget kage har vi brug for?

{\flushright\emph{Hilsen den ikke-kommutative studenterrådgivning}}

\end{minipage}%
\hfill\begin{minipage}[t]{0.47\textwidth}
\vspace{2mm}
\section*{Vejrudsigt}
\textbf{IMF, AU (fra DMI)}: Temperaturer fra 13 til 22 grader og lettere skyet med svag vind fra vest. Der forventes et moderat antal græspollen, og et lavt antal bynkepollen.

\textbf{RUC}: Grasrøg dækker himlen, mens varmen fra vore hjerter føles som i mødrenes skød. Rundkredsen er næsten rund og det sne afvises som etablishementets forsøg på at få et fornøjet liv.

\section*{Fakta om Jylland}
Ifølge den jyske metrik er alt under $25$ km gåafstand.

\section*{Dagens ingeniør}
Hvis du møder en ingenr, så prik hende, og gør grin med at hun faktisk kan tælle.

\section*{Dagens sandsynlighed}
Sandsynligheden for at at få nok søv på Mat-Camp er $0$.

\section*{Dagens opgave}
Som nogen af jer har lært i sandsynlihedsteoriforløbet er der en stor sandsynlighed for at der er mindst to persner på Mat-Camp der har fødselsdag samme dato. 

Vi ved dog ikke hvilke af jer der er de (u)heldige, så derfor opfordrer vi til, at i finder på en smart måde til at finde ud af, hvem der er født samme dag. 

Taberne, der er tvunget til at deles om en fødselsdag, vil vinde et signeret eksemplar af fredagens dagsseddel.

\section*{Dagens bonusopgave}
Prøv at finde billeder af de fastansatte på instituttet for at afgøre på om deres kvindeandel er større end campens andel af kvindelige deltagere.

\end{minipage}

\begin{center}
\includegraphics[width=\linewidth]{social_media.png}
\tiny Randall Munroe, http://xkcd.com/1239/, CC-BY-SA-2.5

\tiny UNF Newz er avisen hvor at ansvarshavende redaktør fralægger sig ethvert ansvar for eventuel plagiering, kaniner, tysk, stavefelj, kaffe, dårlig humor, glemsomhed, katte, store sigmaer, pile, skyer, dårlige oversættelser og alt hvad eventuelle homo sapiens sapiens kunne finde på at holde imod UNF Newz! Dog tager UNF Newz fuld credit og copyright for alle guldkorn, magickort, mus, \TeX, humor, smil, Mortener, kaffe, før-fremtid, ringe og/eller rubik's cube.
\end{center}
\end{minipage}

