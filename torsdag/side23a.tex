\begin{minipage}[t]{120mm}
\vspace{3mm}
\section*{Enjhøringens afkom}
En enhjøring med vinger, eller pegasus med horn, kaldes alicorn.
En alicorn kan kun fødes af et par, hvor faren er pegasus og moren er enhjøring.
Hvis det er omvendt (moren er pegasus og faren er enhjøring), fødes en måneskins hest. Det er en normal hvid hest , bortset fra at hovene og håret er guldfarvet, og indeholder glimmer der opfylder alle ønsker
{\flushright\emph{- kilde Svenjas udgave af google}}

\section*{Dagens citater}

\subsection*{Einstein}
Matematik handler udelukkende om begrebers relation til hinanden, uden hensyn til virkeligheden. 

\subsection*{Storm P.}
Et synonym er et ord, man kan bruge når man ikke kan stave til det man tænkte på først. 

\section*{Produktanmeldelse - Chokoladekiks}
Ved første øjekast ses straks at her virkelig er kredset for detaljerne! Denne ultimative metrik er den eneste målestok, som kan bruges til på en videnskabelig måde at sammenlige andre madvarer. Chokoladekiks (CH) har med sine to velformede kiks, der omslutter den guddommelige midte lavet af det fineste japanske chokolade smeltet og formet mellem lårerne på japanske jomfruer. \\
CH har, som den eneste kendte madvare, egenskaben at vække selv den tungest sovende Mat-Camp deltager op til dåd, det er den ungdommelige kraft, som de japanske jomfruer har lagt i dem, der allerede efter første bid stråler ud af deltagerne og giver dem ny oprejsning så de kan klare dagens strabadser. CH sælges sommetider som ``Prinzen rolle'', da selv de gamle tyske grever kunne finde ud af at værdsætte den kraft de unge japanske piger kunne tilføre CH.\\
CH indeholder ikke nogle løsdele, men er så upassende at de ikke bør serveres til børn under 5 år.

\includegraphics[width=\linewidth]{egern.jpg}
\caption{Dette egern savner sine gulerødder!}
\end{minipage}
