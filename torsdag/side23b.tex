\begin{minipage}[t]{130mm}
\vspace{3mm}
\section*{Bag kulisserne}

Modsat den udbredte holdning om at arrangører går i hi, når de ikke ses af deltagere har vi nu hørt rygter om, at de faktisk også har et liv på andre tidspunkter af året. Vi ved endnu ikke om vi skal tro på dette rygte, men vælger alligevel at beskrive de informationer der foreligger indtil videre.

Øjensynligt har der i dagene inden deltagernes opståen, været heftig aktivitet, både på det faglige, sociale og \censor{kreative} plan blandt campens arrangører. Der er blevet spist, sunget og \censor{grint}.

Der blev drukket en masse \censor{kaffe} og også flere flasker \emph{vand}, mens nogle arrangører blev mere og mere \censor{trippede} og mærkelige.

Arrangørerne \censor{Steffen} og \censor{Tobias} er desuden blevet overhørt mens de \emph{snakkede} i sovelokalet, og der går rygter om at de flere gange siden har \censor{undervist} \censor{deltagerne} i store mængder, \censor{funktioner} \censor{og} \censor{uendelighed}. 

\includegraphics[width=\linewidth]{Mads.jpg}

\section*{Dr. Pjuskebusks Natur Fakta - Stankelben}

Stankelbenet er et insekt af insektfamilien Tipulidae og kan genkendes på deres bemærkelsesværdige hovedform der til forveksling kunne sammenlignes med Fedtmule, hvis man ser bort fra gevækster som følehorn og lignende. 

En anden karakteristika ved stankelbenet er dets, som antydet af navnet, lange tynde ben, men også den lange bagkrop, der primært er hul og benyttes til respiration.

Stankelbenet forveksles til tider med kæmpemyggen, der kan observeres under tiden på den Nordjyske hede. I modsætning til kæmpemyggen, der lever af blodmættede tæger samt andre dyr og mennesker, lever stankelbenet blot af nektar og er således ikke på nogen måde et farligt dyr. Ligesom sølvfisken, der er fast inventar på de fleste badeværelser når natten falder på, behøver man derfor ikke bliver bange eller prøve at klaske den, hvilket desuden ville resultere i en alarm hos Dyrets Værn, hvis stankelbenet var et af de ca. ti millioner stamregistrerede stankelben i Jylland. Som oftest er alarmberedskabet dog allerede rykket ud til en anden hændelse, så sandsynligheden for retslige følger er minimal. "Senere tids mange hændelser har dog rystet Dyrets Værn" lyder en udtalelse fra forbundets talsmand Mark Vorsen, der også har udtrykt at forbundet har planer om en oprustning af alarmberedskabet inden for en overskuelig årrække.

På engelsk kaldes stankelbenet til tider "Daddy Longlegs", hvilket er et navn det deler med flere andre stankel benede dyr, som det dog ikke er beslægtet med. De kan desuden variere i størrelser fra ca. 6 cm til op mod 10 cm, men typisk bliver de ikke så store i Jylland.

En sidste fakta om stankelbenet
Før stankelbenet bliver et stankelben er det en larve og i 1935 blev flere tusinde stankelbenslarver fjernet fra Lord's Criket Ground i London, hvor de åd pletvise huller i græsplænen.

\end{minipage}
